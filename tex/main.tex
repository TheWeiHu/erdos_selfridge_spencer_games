%%%%%%%%%%%%%%%%%%%%%%%%%%%%%%%%%%%%%%%%%
% Lachaise Assignment
% LaTeX Template
% Version 1.0 (26/6/2018)
%
% This template originates from:
% http://www.LaTeXTemplates.com
%
% Authors:
% Marion Lachaise & François Févotte
% Vel (vel@LaTeXTemplates.com)
%
% License:
% CC BY-NC-SA 3.0 (http://creativecommons.org/licenses/by-nc-sa/3.0/)
% 
%%%%%%%%%%%%%%%%%%%%%%%%%%%%%%%%%%%%%%%%%

%----------------------------------------------------------------------------------------
%	PACKAGES AND OTHER DOCUMENT CONFIGURATIONS
%----------------------------------------------------------------------------------------

\documentclass{article}

\input{structure.tex} % Include the file specifying the document structure and custom commands

% Theorems Format
\usepackage{amsthm}

\theoremstyle{plain}
\newtheorem{thm}{Theorem}[section]
\newtheorem{lem}[thm]{Lemma}
\newtheorem{prop}[thm]{Proposition}
\newtheorem*{cor}{Corollary}

\theoremstyle{definition}
\newtheorem{defn}{Definition}[section]
\newtheorem{conj}{Conjecture}[section]
\newtheorem{exmp}{Example}[section]

\theoremstyle{remark}
\newtheorem*{rem}{Remark}
\newtheorem*{note}{Note}

% Floor Function
\newcommand{\floor}[1]{\lfloor #1 \rfloor}

%----------------------------------------------------------------------------------------
%	HEADER
%----------------------------------------------------------------------------------------

\title{CSI 4900: Honours Project Report} % Title of the assignment

\author{Wei Hu\\ \texttt{whu061@uottawa.ca}} % Author name and email address

\date{University of Ottawa --- \today} % University, school and/or department name(s) and a date

%----------------------------------------------------------------------------------------

\begin{document}

\maketitle % Print the title

%----------------------------------------------------------------------------------------
%	INTRODUCTION
%----------------------------------------------------------------------------------------

\section{Introduction} % Unnumbered section

An Erdos-Selfridge-Spencer game consists of an attacker $A$, a defender $D$, and a position $P$. We use $p_{i}$ to represent a piece that is $i$ levels away from the top of the board. In other words, a piece at $p_{0}$, if not destroyed by the defender this turn, will attain tenure and provide a point for the attacker.

\begin{defn}
Let $v$ be a value function, which maps a level $i$ to a real value representing the value of a $p_{i}$ piece. A value function is represented by an array $(w_{0}, w_{1}, w_{2}, ... , w_{k})$ where:
\begin{equation*}
		v(i) = w_{i}.
\end{equation*}
\end{defn}

\begin{defn}
Let $S$ be an array $(a_{0}, a_{1}, a_{2}, ... , a_{k})$ representing a multiset of pieces consisting of $a_{i}$ pieces on the $i$th level. We define the value function applied to $S$ as:
\begin{equation*}
	v(S) = \sum_{i = 0}^k a_{i}v(i).
\end{equation*}
\end{defn}

A defender would apply its value function to the two sets partitioned by the attackers, and it would destroy the set it deems to be more valuable.

\begin{thm}
The optimal value function for the defender is:
% Math equation/formula
\begin{equation}
	v_{*}(i) = \frac{1}{2^{i+1}}.
\end{equation}
\end{thm}

\subsection{Project Summary}

We intend to investigate using reinforcement learning to train attackers to exploit the sub-optimality of biased defenders (nearsighted and farsighted).

%----------------------------------------------------------------------------------------
%	BIASED DEFENDERS
%----------------------------------------------------------------------------------------

\section{Biased Defenders} % Numbered section

We notice that the optimal value function satisfies this property that $v(i) = 2v(i+1)$. By deviating from this equality, we create sub-optimal defenders with biases that either favour pieces that are closer to the top of the board, or those that are at the bottom, depending on the direction of the inequality.

\begin{defn}
A nearsighted (myopic) defender is a defender whose value function satisfies the property, for all defined values of $i$:
\begin{equation}
	v(i) > 2v(i + 1).
\end{equation}
\end{defn}

A nearsighted defender disproportionally values pieces that are close to the top of the board. 

\begin{defn}
A farsighted defender is a defender whose values function satisfies the property, for all defined values of $i$:
\begin{equation}
	v(i) < 2v(i + 1).
\end{equation}
\end{defn}

A farsighted defender disproportionally value pieces that are close to the bottom of the board. \\

\begin{thm}
When playing against an optimal defender, the optimal approach for the attacker is to try to make the value according to $v_{*}$ of the two partitioned sets as close as possible. 
\end{thm}

In order to maximize the number of pieces tenured, it is sufficient to maximize. at each turn, the value according to $v_{*}$ of the surviving pieces. This idea can be proven in two steps: 
\begin{enumerate}
  \item show that against an optimal defender, the best score that can be achieved for starting position $S$ is $\floor{v_{*}(S)}$.
  \item playing according to \textit{Theorem 2.1} achieves that best score. 
\end{enumerate}

%------------------------------------------------

\subsection{Nearsighted (Myopic) Defender}

\begin{thm}
To maximize to real value of the surviving pieces on any round against a nearsighted defender, we follow the following algorithm:
\end{thm}

\begin{center}
	\begin{minipage}{1\linewidth} % Adjust the minipage width to accomodate for the length of algorithm lines
		\begin{algorithm}[H]
			\KwIn{a board position $S$, and the value function of a nearsighted defender $v$}  % Algorithm inputs
			\KwResult{$(S_{1}, S_{2})$, such that $a+b = c + d$} % Algorithm outputs/results
			\medskip
			\If{$\vert b\vert > \vert a\vert$}{
				exchange $a$ and $b$ \;
			}
			$c \leftarrow a + b$ \;
			$z \leftarrow c - a$ \;
			$d \leftarrow b - z$ \;
			{\bf return} $(c,d)$ \;
			\caption{\texttt{Maximizing Real Value Playing Nearsighted Defender}} % Algorithm name
			\label{alg:nearsighted}   % optional label to refer to
		\end{algorithm}
	\end{minipage}
\end{center}

\begin{proof}

\end{proof}


	
%------------------------------------------------

\subsection{Farsighted Defender}

In malesuada ullamcorper urna, sed dapibus diam sollicitudin non. Donec elit odio, accumsan ac nisl a, tempor imperdiet eros. Donec porta tortor eu risus consequat, a pharetra tortor tristique. Morbi sit amet laoreet erat. Morbi et luctus diam, quis porta ipsum. Quisque libero dolor, suscipit id facilisis eget, sodales volutpat dolor. Nullam vulputate interdum aliquam. Mauris id convallis erat, ut vehicula neque. Sed auctor nibh et elit fringilla, nec ultricies dui sollicitudin. Vestibulum vestibulum luctus metus venenatis facilisis. Suspendisse iaculis augue at vehicula ornare. Sed vel eros ut velit fermentum porttitor sed sed massa. Fusce venenatis, metus a rutrum sagittis, enim ex maximus velit, id semper nisi velit eu purus.

\begin{center}
	\begin{minipage}{1\linewidth} % Adjust the minipage width to accomodate for the length of algorithm lines
		\begin{algorithm}[H]
			\KwIn{a board position $P$, and the value function of a nearsighted defender $v$}  % Algorithm inputs
			\KwResult{$(c, d)$, such that $a+b = c + d$} % Algorithm outputs/results
			\medskip
			\If{$\vert b\vert > \vert a\vert$}{
				exchange $a$ and $b$ \;
			}
			$c \leftarrow a + b$ \;
			$z \leftarrow c - a$ \;
			$d \leftarrow b - z$ \;
			{\bf return} $(c,d)$ \;
			\caption{\texttt{Maximizing Real Value Playing Nearsighted Defender}} % Algorithm name
			\label{alg:nearsighted}   % optional label to refer to
		\end{algorithm}
	\end{minipage}
\end{center}

Fusce varius orci ac magna dapibus porttitor. In tempor leo a neque bibendum sollicitudin. Nulla pretium fermentum nisi, eget sodales magna facilisis eu. Praesent aliquet nulla ut bibendum lacinia. Donec vel mauris vulputate, commodo ligula ut, egestas orci. Suspendisse commodo odio sed hendrerit lobortis. Donec finibus eros erat, vel ornare enim mattis et.

Mauris interdum porttitor fringilla. Proin tincidunt sodales leo at ornare. Donec tempus magna non mauris gravida luctus. Cras vitae arcu vitae mauris eleifend scelerisque. Nam sem sapien, vulputate nec felis eu, blandit convallis risus. Pellentesque sollicitudin venenatis tincidunt. In et ipsum libero. Nullam tempor ligula a massa convallis pellentesque.

%----------------------------------------------------------------------------------------

\end{document}

%----------------------------------------------------------------------------------------
%	TEMPLATE ELEMENTS
%----------------------------------------------------------------------------------------

% Numbered question, with subquestions in an enumerate environment
%\begin{question}
%	Quisque ullamcorper placerat ipsum. Cras nibh. Morbi vel justo vitae lacus tincidunt ultrices. Lorem ipsum dolor sit amet, consectetuer adipiscing elit.
%
%	% Subquestions numbered with letters
%	\begin{enumerate}[(a)]
%		\item Do this.
%		\item Do that.
%		\item Do something else.
%	\end{enumerate}
%\end{question}

% Numbered question, with an optional title
%\begin{question}[\itshape (with optional title)]
%	In congue risus leo, in gravida enim viverra id. Donec eros mauris, bibendum vel dui at, tempor commodo augue. In vel lobortis lacus. Nam ornare ullamcorper mauris vel molestie. Maecenas vehicula ornare turpis, vitae fringilla orci consectetur vel. Nam pulvinar justo nec neque egestas tristique. Donec ac dolor at libero congue varius sed vitae lectus. Donec et tristique nulla, sit amet scelerisque orci. Maecenas a vestibulum lectus, vitae gravida nulla. Proin eget volutpat orci. Morbi eu aliquet turpis. Vivamus molestie urna quis tempor tristique. Proin hendrerit sem nec tempor sollicitudin.
%\end{question}

% File contents
%\begin{file}[hello.py]
%\begin{lstlisting}[language=Python]
%#! /usr/bin/python
%
%import sys
%sys.stdout.write("Hello World!\n")
%\end{lstlisting}
%\end{file}

% Warning text, with a custom title
%\begin{warn}[Notice:]
%  In congue risus leo, in gravida enim viverra id. Donec eros mauris, bibendum vel dui at, tempor commodo augue. In vel lobortis lacus. Nam ornare ullamcorper mauris vel molestie. Maecenas vehicula ornare turpis, vitae fringilla orci consectetur vel. Nam pulvinar justo nec neque egestas tristique. Donec ac dolor at libero congue varius sed vitae lectus. Donec et tristique nulla, sit amet scelerisque orci. Maecenas a vestibulum lectus, vitae gravida nulla. Proin eget volutpat orci. Morbi eu aliquet turpis. Vivamus molestie urna quis tempor tristique. Proin hendrerit sem nec tempor sollicitudin.
%\end{warn}

% Command-line "screenshot"
%\begin{commandline}
%	\begin{verbatim}
%		$ chmod +x hello.py
%		$ ./hello.py
%
%		Hello World!
%	\end{verbatim}
%\end{commandline}